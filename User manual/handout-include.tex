%%% include file for lectures

%% packages
\usepackage{etex}
\usepackage{amsmath}
\usepackage{amssymb}
\usepackage{amsthm}
\usepackage{amsfonts}
\usepackage{epsfig}
\usepackage{latexsym}
\usepackage{url}
\usepackage{ifthen}
\usepackage{comment}
\usepackage{enumerate}
\usepackage{multirow}
\usepackage[english]{babel}
\usepackage{texnansi}
\usepackage{textcomp}
\usepackage{color}
\usepackage{calc}
\usepackage[normalem]{ulem}
\usepackage{array}
\usepackage{booktabs}
\usepackage[margin=10pt,font=small,labelfont=bf]{caption}
\usepackage[hypertexnames=false,hyperfootnotes=false,bookmarks=false]{hyperref}
\usepackage{sectsty}

%% spacing
\usepackage{setspace}
\usepackage[margin=1in]{geometry}

% boolean to use Lucida Bright as the font
\provideboolean{lucidabr}
% uncomment to use Lucida
%\setboolean{lucidabr}{true}

%% pick the font
\ifthenelse{\boolean{lucidabr}}{%
  % Lucida Bright
  \usepackage{lucidabr}
}{%
  % following two use Helvetica Neue for text, Latin Modern for math
  \usepackage{lmodern}
 %%%% \usepackage{gtamachelveticaneue}
}

%% customize section titles
\allsectionsfont{\large\sffamily}
\makeatletter
\def\@seccntformat#1{\csname the#1\endcsname.\quad}
\makeatother

%% tikz/pgf packages
\usepackage{tikz}
\usepackage{pgfpages}
\usepackage{pgfplots}
\usepackage{pgfplotstable}
\usetikzlibrary{arrows,decorations,patterns,trees,matrix,calc,shapes}
%this is obsolete, but useful for compatibility
%\usetikzlibrary{snakes}

%% useful math macros
\newcommand{\field}[1]{\mathbb{#1}}
\newcommand{\N}{\field{N}} % natural numbers
\newcommand{\R}{\field{R}} % real numbers
\renewcommand{\Re}{\R} % real numbers
\newcommand{\Z}{\field{Z}} % integers
\newcommand{\1}{\mathbf{1}} % vector of all 1's
\newcommand{\0}{\mathbf{0}} % vector of all 0's
\newcommand{\I}[1]{\mathbf{1}_{\left\{#1\right\}}} % indicator function
\newcommand{\Inb}[1]{I_{#1}} % indicator function, no brackets
\newcommand{\tends}{{\rightarrow}} % arrow for limits
\newcommand{\ra}{{\rightarrow}} % abbreviation for right arrow
\newcommand{\PR}{\mathsf{P}} % probability
\newcommand{\E}{\mathsf{E}} % expectation
\newcommand{\defeq}{\triangleq}
\newcommand{\subjectto}{\text{subject to}} % subject to
\newcommand{\ip}[2]{\langle #1, #2\rangle} % inner product

%% math operators
\DeclareMathOperator{\PV}{\text{PV}}
\DeclareMathOperator{\inter}{\textbf{int\,}}
\DeclareMathOperator{\relint}{\textbf{relint\,}}
\DeclareMathOperator{\dom}{\textbf{dom\,}}
\DeclareMathOperator{\cl}{\textbf{cl\,}}
\DeclareMathOperator{\conv}{\textbf{conv\,}}
\DeclareMathOperator{\card}{\textbf{card}}
\DeclareMathOperator{\aff}{\textbf{aff\,}}
\DeclareMathOperator{\var}{\text{Var}}
\DeclareMathOperator{\Var}{\text{Var}}
\DeclareMathOperator{\cov}{\text{Cov}}
\DeclareMathOperator{\sgn}{\text{sgn}}
\DeclareMathOperator{\TV}{\text{TV}}
\DeclareMathOperator{\BV}{\mathrm{BV}}
\DeclareMathOperator{\NBV}{\mathrm{NBV}}
\DeclareMathOperator*{\argmin}{\text{argmin}}
\DeclareMathOperator*{\argmax}{\text{argmax}}
\newcommand\minimize{\mathop{\text{minimize}}\limits}
\newcommand\maximize{\mathop{\text{maximize}}\limits}
\newcommand{\im}{\text{im\,}}
% redfine these to get the font right
\renewcommand{\ker}{\text{ker\,}}
\renewcommand\max{\mathop{\text{max}}\limits}
\renewcommand\min{\mathop{\text{min}}\limits}
\renewcommand\lim{\mathop{\text{lim}}\limits}
\renewcommand\log{\mathop{\text{log}}\limits}
\renewcommand\exp{\mathop{\text{exp}}\limits}
\renewcommand\sup{\mathop{\text{sup}}\limits}
\renewcommand\inf{\mathop{\text{inf}}\limits}
\renewcommand\sin{\mathop{\text{sin}}\limits}
\renewcommand\cos{\mathop{\text{cos}}\limits}
\renewcommand\tan{\mathop{\text{tan}}\limits}

%% some caligraphic symbols
\newcommand{\Ascr}{{\mathcal A}}
\newcommand{\Bscr}{{\mathcal B}}
\newcommand{\Cscr}{{\mathcal C}}
\newcommand{\Dscr}{{\mathcal D}}
\newcommand{\Escr}{{\mathcal E}}
\newcommand{\Fscr}{{\mathcal F}}
\newcommand{\Gscr}{{\mathcal G}}
\newcommand{\Hscr}{{\mathcal H}}
\newcommand{\Iscr}{{\mathcal I}}
\newcommand{\Jscr}{{\mathcal J}}
\newcommand{\Kscr}{{\mathcal K}}
\newcommand{\Lscr}{{\mathcal L}}
\newcommand{\Mscr}{{\mathcal M}}
\newcommand{\Nscr}{{\mathcal N}}
\newcommand{\Oscr}{{\mathcal O}}
\newcommand{\Pscr}{{\mathcal P}}
\newcommand{\Qscr}{{\mathcal Q}}
\newcommand{\Rscr}{{\mathcal R}}
\newcommand{\Sscr}{{\mathcal S}}
\newcommand{\Tscr}{{\mathcal T}}
\newcommand{\Uscr}{{\mathcal U}}
\newcommand{\Vscr}{{\mathcal V}}
\newcommand{\Wscr}{{\mathcal W}}
\newcommand{\Xscr}{{\mathcal X}}
\newcommand{\Yscr}{{\mathcal Y}}
\newcommand{\Zscr}{{\mathcal Z}}

%% abstract, table, figure names
\renewcommand{\figurename}{\normalfont\sffamily\bfseries Figure}
\renewcommand{\tablename}{\normalfont\sffamily\bfseries Table}

%% other customizations
\newcommand{\emailhref}[1]{\href{mailto:#1}{\tt #1}} % hyperlinked email address

%% always use 'stealth' arrows
\tikzstyle{every picture} += [>=stealth]

%% styles for figures
\tikzset{axis/.style={semithick, line join=miter}}

%% setup some layers
\pgfdeclarelayer{background}
\pgfdeclarelayer{foreground}
\pgfsetlayers{background,main,foreground}

%% table defaults
\pgfplotstableset{%
  font=\small,
  every head row/.style={before row=\toprule[1pt], after row=\midrule},
  every last row/.style={after row=\bottomrule[1pt]}}

%% pstricks ncline functionality
\newcommand{\tikznode}[2]{%
  \tikz[remember picture,baseline]%
  \node[anchor=base,inner sep=0pt] (#1) {#2};%
}
\newcommand{\tikznodefull}[3]{%
  \tikz[remember picture,baseline]%
  \node[anchor=base,inner sep=0pt,#2] (#1) {#3};%
}
\newcommand{\tikzcoord}[1]{%
  \tikz[remember picture,overlay,baseline=-0.5ex]%
  \node [coordinate] (#1) {};%
}
\newcommand{\tikzline}[3]{%
  \tikz[remember picture,overlay,baseline]%
    \draw [#1] (#2) to (#3);%
}

% a colored box around text
\newcommand{\tikznodebox}[2]{%
  \tikz[baseline]%
  \node[anchor=base,inner sep=0pt,#1] {#2};
}

%% spacing
\newcommand{\skipline}{\vspace{\baselineskip}}

%% macro for course name
\def\coursename#1{%
  \def\ctemp{#1}%
  \ifx\ctemp\@empty
  \def\insertcoursename{}
  \else
  \def\insertcoursename{\ignorespaces#1}
  \fi
}
\coursename{}

%% macro for the title

\makeatletter
\renewcommand{\maketitle}{\par%
  \newpage
  \global\@topnum\z@
  \null
  \vspace*{-2.5cm}
  \vskip 2em%
  \includegraphics[height=0.3in]{cbs-logo-horizontal}
  \hfill
  \parbox[b]{1.95in}{\noindent\bfseries\sffamily \insertcoursename \\
    \hskip 200pt}% \@date}
  \begin{center}
    {\large\bfseries\sffamily \@title}
  \end{center}
}
\makeatother

%% indentation, etc.
\setlength{\parindent}{0pt}
\setlength{\parskip}{10pt}

%% course name
%\coursename{B6101 Business Analytics}

%% date
%\date{Spring 2014}

%% author
%\author{Professor: Ciamac Moallemi}

%%% Local Variables:
%%% mode: latex
%%% TeX-master: 'shared
%%% End:
